\documentclass{article}

\begin{document}

\begin{Large}

Quick game logic documentation.

\end{Large}

\begin{itemize}

\item draw_grid function is responsible for drawing the grid of tic-tac-toe

It takes "frame" as an argument, which is the current frame in the video stream

\item draw_o function is responsible for drawing O's on the screen, it takes as an argument the frame, a bool called player, if player is true then it's player1, if it is false then it is player2. center_x and center_y are coordinates returned by the YOLO module

firstly it iterates over each element from the pin list which is declared on line 160, it is a 2D list which stores the coordinates of each square in the grid, each square is a list which has 3 elements, its x position, y position and an integer that equals 0 by default.

For each box we check if the coordinates received are in between the boundries of the square, if thats the case, the last element in the list that represents the square/box gets set to 2 which stands for player 2 unless it is equal to 1 which stands for player 1, we assumed that player 1 will play as x and player 2 will play as O.

After that we check if the box is occupied by player 2, if thats the case, we print the circle

Finally we call the win_check function in order to check whether the game ended or no

\item draw_x function, works like the draw_o function except for the part responisble for drawing the X, which is drawn using lines

\item print_text function, it is used to quickly print text

\item print_timer function, it is a function that prints the current time of the game, it takes "start" as an argument which is initialized on line 198, it stores the time at the start of the game. each time the function is called it calculates the current time then prints it

\item count_down, the main purpose of this function is to control when does the YOLO module reads a frame from the video, it is by default set to 2 seconds

\item turn function, it is used to switch turns between the players but the players are not obligated to follow it. But it is still important because it calls the count_down function and takes a bool return from it indicating whether the countdown ended or not, if it ended, the module will read a frame as will be clear later in the code.

\item end, this function is called to end the game

\item win_check, it checks if the game ended by checking if the squares of a row or a column are occupied by the same player, then checks if the squares of a diagonal are occupied by the same player

\item line138 to line165 are self explainatory

\item predict function, it is the function that is actually responsible for interfacing with the module and activating it, it returns the coordinates detected by the module and a bool representing the player, if the module does not detect anything, the coordinates will be returned as (-1,-1)

\item line 200 marks the start of the main game loop, it has comments that will help explain what is going on

\item line 218 is responsible for actually displaying the frame, and the rest is responsible for keeping the window open until "q" is pressed

\end{itemize}

\end{document}
